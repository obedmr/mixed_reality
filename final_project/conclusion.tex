\section{Conclusion}

\begin{enumerate}
\item Google Geofences are an effective solution to the user localization problem: they are easy to use, configuration is made within the Android device, and Google Play Services takes care of the monitoring and entry/exit events.

\item Another solution for the Geofence localization problem is the Gimbal frameworkGimbal is more effective,  in my experience, tan Google APIs, but it requires more configuration steps, as well as a Web server to interact with the App, and two more services to monitor the user location, and to receive notifications. Gimbal also has a cost in its paid version.

\item Geofences will be present in the future of the Mobile Augmented Reality. Every day the need to Access the user location is more frequent, and the possibility to offer accurate information to the user is a big business opportunity in áreas such as advertisement, marketing, and tourism.
\end{enumerate}

\textit{Adrian Garcia Betancourt} \\\\

\begin{enumerate}
  \item Geofencing in combination with Augmented Reality provide a promising 
        opportunity area in Tourism area 
  \item In less than 10 years will be able to see this kind of AR approaches
        embeded in mobile devices
  \item When developing AR applications for mobile devices, it's important
        to consider the best practices for AR in Tourism that consists in 
        developing it in such way that the User Experience is improved and 
        avoid the privacy intrucion
\end{enumerate}

\textit{Obed N Munoz Reynoso}
