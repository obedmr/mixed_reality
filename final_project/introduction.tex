\section{Introduction}
% no \IEEEPARstart
With the enormous bum of powerful smarth devices like smartphones, wearables and more
recently the Google Glasses, th Augmented Reality (AR) is becoming the new trend topic
in the world. This combination of virtual and physical world information has not been
utilized at all in the Tourism field. \\

Augmented Reality (AR) is providing an extension of key visual and non-visual elements
that are improving user experience in her/his interaction with the technology. 

This paper explains how Google Geofences Java Framework can improve the user experience
on touristic places. Google provides an innovative API that can provide 


\subsection{AR trends}
Augmented Reality solutions on this

\subsection{Related Projects}
Talk about Prisma Prject

gimbal


\subsubsection{Research}
- tourism in Guadalajara

- Tourism in Mexico Destinations

- Augmented Reality best practices

\subsection{Hardware Requirements}
Nowadays that smartphones are being used by a 63.5\% of  the 4.55 billion of mobile phone users[1], it's a great 
opportunity to introduce mobile applications that take advantage of mobile technology like 3G/4G and GPS. \\

In order to run the Travelex application, it's necesary to have a smartphone with the following specs:

\begin{itemize}
  \item Operative System: Android (15 to latest versions)
  \item GPS enabled  
  \item Internet connectivity (WiFi or 3G/4G)
\end{itemize}


\subsubsection{Justification}
Given the current number of Android based smartphones (52.1\%)[2], it's a good start point for this application to be 
released. \\
It's also easier to implement the Google Geofences on an Android phone because the code can more natively implemented.

\subsection{Programming Techniques}
Android is an Operative System is mainly based in Java programming language. The Travelex application was fully developed
on Java with the implementation of RESTFul calls for the Google Geofencing API and some SQL queries for the database
interaction. \\

Below are the required development tools and frameworks:

\begin{itemize}
  \item Programming Language: Java
  \item Database: SQLite
  \item API: RESTFul with Google Geofences
  \item Frameworks: Google Geofences, Android SDK
  \item IDE: Android Studio
\end{itemize}

\subsubsection{Justification}


\subsubsection{Comparison}
- Comparison between Google Geofences/Gimbal
