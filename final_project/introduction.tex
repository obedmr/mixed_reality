\section{Introduction}
% no \IEEEPARstart
With the enormous bum of powerful smarth devices like smartphones, wearables and more
recently the Google Glasses, th Augmented Reality (AR) is becoming the new trend topic
in the world. This combination of virtual and physical world information has not been
utilized at all in the Tourism field. \\

Augmented Reality (AR) is providing an extension of key visual and non-visual elements
that are improving user experience in her/his interaction with the technology. \\

Geo-fencing is a technique for defining a virtual perimeter or real-world geographical 
area. A geo-fence could be dynamically generated—as in a radius around a store or point location. Or a geo-fence can be a predefined set of boundaries, like school attendance zones or neighborhood boundaries. \\

Custom-digitized geofences have been in use since 2004 for multiple online mapping applications since their development by Dr. Vinay Rawlani at the University of Missouri-Columbia. \\

When the location-aware device of a location-based service (LBS) user enters or exits a geo-fence, the device receives a generated notification. This notification might contain information about the location of the device. The geofence notice might be sent to a mobile telephone or an email account.\\

With an increasing popularity of mobile advertising, geofencing has been employed to distribute location specific ads to customers on their mobile devices. This paper explains how we implemented the Google GeoFences API for locating a tourist and provide him/her automatic alerts of the near places that may be interesting to visit.



\subsection{AR trends}
There are many opportunities where Augmented Reality can be applied and achieve 
interesting benefits. Talking about the Tourism sector, it has been generating a very
lucrative approach for many corporations. For example, hotels are always looking for 
improving the user experience, for that there are a solution for enabling virtual
tours to hotel guests as an entertainment. \\

Also, If we talk about education, there are some investigations in the museums and 
ancient cities that are providing Augmented Reality for simulating un-existing scenarios
or re-creating an specific environment as it was many years ago.\\


\subsection{Related Projects}
Regarding projects that are allowing a better user experience with the use of Augmented Reality, there are many 
approaches around the world. Some of them combine the Virtual and Augmented Reality in order to provide altered
landscapes and in some case it's providing a new virtual environment that is based in the focused one but in its 
recreated version of many years ago. \\

The PRISMA[7] project is an example of this AR applications. 
The main objective of PRISMA is the implementation of augmented binoculars, that combine the needs of tourists
in real environments and Augmented Reality technologies. The use of these technologies will allow the users retrieving
personalized and interactive multimodal information about monuments and historical buildings of a city. \\

If we focus in technologies that are being used for Geofencing, the Gimbal[5] project  provides geofencing, secure 
proximity beacons, location-based messaging, analytics and consumer privacy controls. 

\subsubsection{Research}
First of all we wanted to focus on the best practices of Augmented Reality that is applied to Tourism[8]:

\begin{enumerate}
  \item An Enhanced Booking Experience
  \item Museum Interactivity
  \item AR Browsers in the Destination
  \item Responsive Experience Through Gaming
  \item Services in the Restaurant
  \item Re-living Historic Life and Events
  \item The Hotel Experience
  \item Transportation
  \item Accessibility and Translation
  \item Participative Destination Management
\end{enumerate}

The Travelex application has been designed to satisfy most of the best practices for AR applications in Tourism.
It is improving the user experience at least in the last 3 points because it will provide assistence on finding 
the best places to visit. \\

According to the Secretary of Tourism in Mexico, in the last 5 years, there has been a considerable economic growth
in tourism sector. Below the 2014 key results[6]:\\

\begin{itemize}
  \item Income of foreign exchange from international visitors: 12,037.9 millions of USD (+17.4\%)
  \item Arrival of international visitors: 21,112.8 thousands of people (+18.2\%)
  \item Medium spending per international tourist inside the country: 861.7 (+10.5\%)
\end{itemize}

The shown percentage is the comparison with the last year (2013). It means that there is a big opportunity to grow 
in the Tourism sector with the enablement the Augmented Reality technologies in order to atract more tourist to
visit mexican amazing destinations.


\subsection{Hardware Requirements}
Nowadays that smartphones are being used by a 63.5\% of  the 4.55 billion of mobile phone users[1], it's a great 
opportunity to introduce mobile applications that take advantage of mobile technology like 3G/4G and GPS. \\

In order to run the Travelex application, it's necesary to have a smartphone with the following specs:

\begin{itemize}
  \item Operative System: Android (15 to latest versions)
  \item GPS enabled  
  \item Internet connectivity (WiFi or 3G/4G)
\end{itemize}


\subsubsection{Justification}
Given the current number of Android based smartphones (52.1\%)[2], it's a good start point for this application to be 
released. \\
It's also easier to implement the Google Geofences on an Android phone because the code can more natively implemented.

\subsection{Programming Techniques}
Android is an Operative System is mainly based in Java programming language. The Travelex application was fully developed
on Java with the implementation of RESTFul calls for the Google Geofencing API and some SQL queries for the database
interaction. \\

Below are the required development tools and frameworks:

\begin{itemize}
  \item Programming Language: Java
  \item Database: SQLite
  \item API: RESTFul with Google Geofences
  \item Frameworks: Google Geofences, Android SDK
  \item IDE: Android Studio
\end{itemize}

\subsubsection{Justification}
The above technologies where choosen because there is a vast amount of 
documentation and there is a big community of developers that could help
us in case of a show stopper may appear in the progress of the project.

\subsubsection{Technologies Comparisons}
When choosing the mobile Operative System for implementing the first release
of Travelex, we choose Android becuase is the mobile OS that is being more 
used and it is the one the has the more accesible prices for smartphones.
Windows and iOS based phones would be the next target for a future release. \\

Google Geofences vs Gimbal \\
The Google Goefences provides more documentation and free access for starting
developing a POC Android application. Gimbal seems to provide more features 
that at this moment are not relevant to the project and it also cost more 
money to implement.
