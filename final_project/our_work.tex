\section{Our Work}


\subsection{Identifying places of Interest}
Based on the top 10 touristic places in Mexico shown in the TripAdvisor website[3]: \\

\begin{itemize}
  \item Playa del Carmen
  \item Puerto Vallarta
  \item Mexico City
  \item Cabo San Lucas
  \item Zihuatanejo
  \item Tulum
  \item Oaxaca
  \item San Miguel de Allende
  \item Puerto Escondido
  \item Acapulco
\end{itemize}


In order to provide the first proof of concept (POC), we started in our city; Guadalajara, which is also a 
touristic city.


\subsection{Google Geofences Implementation}
A geofence defines a region of interest. For example, this may be a polygon outside of which a particular set of entities is not expected to stray. Geofences apply to particular collections.\\

Each geofence has a number of properties [4]:

\begin{itemize}
  \item ID: An opaque string used to refer to the geofence in calls to various methods. The ID for a geofence is assigned by the API at creation time. 

  \item Name: A user-defined string describing the geofence.

  \item Collection IDs: Zero or more IDs of collections to which the geofence applies. 

  \item Polygon: A polygon specifying the geofence's region. Crumbs recorded for entities belonging to one of the geofence's collections that fall within this polygon make the geofence active. This is required. 
\end{itemize}

For the  Travelex first approach the work-flow consists in the following 4 steps:\\

\begin{itemize}
  \item The application opens the  SQLite database file where it obtains the pre-defined GeoFences
  \item GeoFences information is sent to Google Geofencing API through a RESTFul API to be created and monitored
  \item A BroadcastReceiver service is started to receive transactions from Google API when it identifies when
        the users is near to a registered touristic place.
  \item The BroadcastReceiver service is started every time the phone is booted in order to let the Google API the phone's
        current location
\end{itemize}

IMAGE Flow Chart\\

NOTE: In our first testing approach, geofences were predefined and loaded at installation time in the SQLite 
database file.

\subsection{Project Schedule}
Gant Diagram

\subsubsection{Personnel Needs}
How many developers, testers (internals and externals)
(national/internationals)

\subsection{Impact of the project}
- Helping the tourists
* Economy perspective
- Improving local business 
* Goverment perspective
- Generating more jobs

FODA diagram



